\documentclass{beamer} % classe di documenti per poter creare le presentazioni che in questo linguaggio vengono definite come frame. 
\usepackage{graphicx} % Package necessario per utilizzare immagini
\usepackage{listings} % Pacchetto che può essere utilizzato per caricare degli algoritmi direttamente dagli script.
\usepackage{multicol}
\usepackage{url}

\definecolor{mygreen}{rgb}{0,0.6,0}


\usetheme{AnnArbor}
\usecolortheme{seahorse}

\title{Variazione della superficie del Lago Laceno}
\author{Vincenzo Busiello}
\date{19 Giugno 2024}

\begin{document}

\maketitle

\AtBeginSection[]
{
\begin{frame}
\frametitle{Outline}    
    \tableofcontents[currentsection]
\end{frame}
}

\section{Lago Laceno}

        \begin{frame}{Lago Laceno}
            \begin{multicols}{2}
                \begin{itemize}
                    \item {\scriptsize 1100 m s.l.m.} 
                    \item {\scriptsize Bagnoli Irpino (AV)}
                    \item {\scriptsize Parco Regionale dei Monti Picentini.}
                \end{itemize}
            \columnbreak
                \begin{center}
                    \includegraphics[width=0.40\textwidth]{LagoLaceno.jpg}
                \end{center}
            \end{multicols}
        \end{frame}

        \begin{frame}{Sempre più piccolo...}
            \begin{multicols}{2}
                \begin{itemize}
                    \item {\scriptsize Bonifiche all'inizio del secolo scorso} 
                    \item {\scriptsize Terremoto dell'Irpinia del 1980}
                    \item {\scriptsize Eventi di siccità sempre più frequenti}
                \end{itemize}
            \columnbreak
                \begin{center}
                    \includegraphics[width=0.40\textwidth]{Lago-Laceno-33.jpg}
                \end{center}
            \end{multicols}
        \end{frame}

        \begin{frame}{Perché questo progetto?}
            \begin{multicols}{2}                
                \begin{center}
                    \includegraphics[width=0.40\textwidth]{homer.png}
                \end{center}
            \columnbreak
                \begin{itemize}
                    \item La superficie del lago è cambiata negli ultimi anni?
                    \item C'è una variazione stagionale? Se sì, di quanto?
                    \item La variazione stagionale è sempre presente?
                \end{itemize}
            \end{multicols}
        \end{frame}


\section{Dati e Codice}


        \begin{frame}{Dati}
            \begin{center}
            Le immagini utilizzate per questo progetto sono state scaricate dal sito \url{https://browser.dataspace.copernicus.eu} utilizzando i seguenti criteri:
                \begin{itemize}
                    \item E' stata selezionata un'area che comprende il lago oggetto d'esame
                    \item Date di confronto: estate - inverno dal 2017 al 2020
                    \item Copertura di nuvole minore del 10\%
                    \item Minor copertura di neve
                    \item Download delle bande 2,3,4 e 8 in formato .tiff 16 bit
                \end{itemize}
            \end{center}
        \end{frame}

        \begin{frame}{Esempio del sito di studio}
                \begin{multicols}{2}
                    \begin{center}
                        \includegraphics[width=0.25\textwidth]{es2017.jpg}\\
                        \caption{{\scriptsize Estate 2017}}\\
                \columnbreak
                        \includegraphics[width=0.25\textwidth]{in2017.jpg}\\
                        \caption{{\scriptsize Inverno 2017}}\\
                    \end{center}
                \end{multicols}
        \end{frame}
        
        \begin{frame}{Normalized Difference Water Index}
            \begin{equation}
                NDWI = \frac{(GREEN - NIR)}{(GREEN + NIR)}
            \end{equation}
        \end{frame}
        
        \begin{frame}{Pacchetti}
            \begin{itemize}
                \item    \texttt{library(terra)} 
                \item    \texttt{library(imageRy)} 
                \item    \texttt{library(viridis)}
                \item    \texttt{library(ggplot2)} 
                \item    \texttt{library(patchwork)}
            \end{itemize}
        \end{frame}
        
            \begin{frame}{Funzioni}
                \begin{multicols}{2}
                    \begin{itemize}
                        \item    \texttt{setwd()} 
                        \item    \texttt{rast()} 
                        \item    \texttt{c()}
                        \item    \texttt{par()} 
                        \item    \texttt{im.plotRGB()}
                        \item    \texttt{plot()}
                        \item    \texttt{im.classify()}
                        \item    \texttt{ncell()}
                        \item    \texttt{freq()}
                \columnbreak
                        \item    \texttt{data.frame()}
                        \item    \texttt{View()}
                        \item    \texttt{ggplot()}
                        \item    \texttt{aes()}
                        \item    \texttt{geom_boxplot()}
                        \item    \texttt{ylim()}
                        \item    \texttt{xlab()}
                        \item    \texttt{ylab()}
                        \item    \texttt{ggtitle()}
                    \end{itemize}
                \end{multicols}
        \end{frame}
        
        \begin{frame}{TrueColor - Codice}
            \lstinputlisting[language=R, firstline=82, lastline=90, commentstyle=\color{mygreen}, basicstyle=\footnotesize]{ScriptEsame.R}
        \end{frame}

        \begin{frame}{TrueColor - Immagine}
            \begin{center}
                \includegraphics[width=0.5\textwidth]{TrueColor.png}
            \end{center}
        \end{frame}

        \begin{frame}{FalseColor - Codice}
            \lstinputlisting[language=R, firstline=94, lastline=102, commentstyle=\color{mygreen}, basicstyle=\footnotesize]{ScriptEsame.R}
        \end{frame}
        
        \begin{frame}{FalseColor - Immagine}
            \begin{center}
                \includegraphics[width=0.5\textwidth]{FalseColor.png}
            \end{center}
        \end{frame}

        \begin{frame}{NDWI - Codice}
            \lstinputlisting[language=R, firstline=111, lastline=120, commentstyle=\color{mygreen}, basicstyle=\footnotesize]{ScriptEsame.R}
        \end{frame}

        \begin{frame}{NDWI - Immagine}
            \begin{center}
                \includegraphics[width=0.5\textwidth]{ViridisNDWI.png}
            \end{center}
        \end{frame}

        \begin{frame}{Classificazione - Codice}
            \lstinputlisting[language=R, firstline=180, lastline=188, commentstyle=\color{mygreen}, basicstyle=\tiny]{ScriptEsame.R}
        \end{frame}

        \begin{frame}{Classificazione - Immagine}
            \begin{center}
                \includegraphics[width=0.5\textwidth]{ClassificazioneDEF.png}
            \end{center}
        \end{frame}

        \begin{frame}{Percentuali di Copertura Divise per Classi}
            \begin{center}
                \includegraphics[width=0.5\textwidth]{tabellaLac.png}
            \end{center} 
        \end{frame}
        
        \begin{frame}{Boxplot Superfici Classificate}
            \begin{center}
                \includegraphics[width=1\textwidth]{boxplot.png}
            \end{center}
        \end{frame}

        \begin{frame}{Boxplot Superficie del Lago}
            \begin{center}
                \includegraphics[width=0.5\textwidth]{GrafAcqua.png}
            \end{center}
        \end{frame}

\section{Conclusioni}

        \begin{frame}{Conclusioni}
            \begin{itemize}
                \item La superficie del lago varia nel periodo 2017-2020. In questi anni lo specchio d'acqua ha ridotto le dimensioni nel periodo estivo mentre nel periodo invernale non sembra seguire un andamento preciso. 
                \item Le diverse dimensioni del lago potrebbero essere legate ai livelli di precipitazione. Queste appaiono progressivamente minori per periodi estivi; le dimensioni invernali sembrano essere legate alle precipitazioni dell'anno precedente. 
            \end{itemize}
        \end{frame}

                \begin{frame}{Conclusioni}
            \begin{itemize}
                \item Invece, nel 2018 vi è una differenza sostanziale rispetto agli altri anni: la superficie del lago è minore nella stagione invernale rispetto alla stagione estiva; confrontando temperature e precipitazioni medie pare non ci sia alcun legame con tale situazione.
                \item Sarebbbe interessante vedere i trend del lago negli anni precedenti e postumi al periodo preso in esame per confermare o meno il trend. 
            \end{itemize}
        \end{frame}
        
\section{Sitografia}
        \begin{frame}{Sitografia}
            \begin{itemize}
                \item \url{https://browser.dataspace.copernicus.eu} \\
                \item \url{https://centrofunzionale.regione.campania.it/#/pages/sensori/archivio-pluviometrici} \\
                \item \url{https://en.wikipedia.org/wiki/Normalized_difference_water_index}\\
            \end{itemize}  
        \end{frame}

\section{Ringraziamenti}
        \begin{frame}{Ringraziamenti}
            \begin{center}
                {\huge Grazie per l'attenzione!}\\
                \bigskip
                \bigskip
                \bigskip
                \bigskip
                \bigskip
                \bigskip
                \bigskip
                \bigskip
                \url{https://github.com/VincenzoBusiello}
            \end{center}
        \end{frame}


\end{document}

\documentclass{beamer} % classe di documenti per poter creare le presentazioni che in questo linguaggio vengono definite come frame. 
\usepackage{graphicx} % Package necessario per utilizzare immagini
\usepackage{listings} % Pacchetto che può essere utilizzato per caricare degli algoritmi direttamente dagli script.
\usetheme{Dresden}
\usecolortheme{sidebartab}

\title{Variazione della superficie del Lago Laceno}
\author{Vincenzo Busiello}
\date{19 Giugno 2024}

\begin{document}

\maketitle

\AtBeginSection[]
{
\begin{frame}
\frametitle{Outline}    
    \tableofcontents[currentsection]
\end{frame}
}

\section{Lago Laceno}

        \begin{frame}{Lago Laceno}
            \begin{itemize}
                \item Lago che nasce sull'omonimo altopiano a 1100 m s.l.m.
                \item Alimentato dal 
                \item Text visible on slide 3
                \item Text visible on slide 4
            \end{itemize}
        \end{frame}

        \begin{frame}{Obiettivi di progetto}
            inserisco il testo qui
        \end{frame}

        \begin{frame}{Dati}
            inserisco il testo qui
        \end{frame}

\section{Codice e Dati}

        \begin{frame}{Materiali}
            inserisco il testo qui
        \end{frame}
        
        \begin{frame}{Materiali}
            inserisco il testo qui
        \end{frame}
        
        \begin{frame}{Materiali}
            inserisco il testo qui
        \end{frame}
        
        \begin{frame}{Materiali}
            inserisco il testo qui
        \end{frame}

\section{Risultati}

        \begin{frame}{Risultati}
            \begin{equation}
                \sigma =\sqrt{\frac{\displaystyle\sum_{i=1}^{N}{(x - \mu)^2}}{N}}
            \end{equation}
        \end{frame}

        \begin{frame}{The code!}
            %\lstinputlisting[language=R]{} % si specifica il linguaggio di programmazione utilizzato all'interno dello script e successivamente i passa come argomento della funzione il nome dello script che dovrà essere caricato su overleaf tramite il tasto carica
        \end{frame}

\section{Conclusioni}

        \begin{frame}{Conclusioni}
            inserisco il testo qui
        \end{frame}

        \begin{frame}{Conclusioni}
            inserisco il testo qui
        \end{frame}

        \begin{frame}{Conclusioni}
            inserisco il testo qui
        \end{frame}

        \begin{frame}{Conclusioni}
            inserisco il testo qui
        \end{frame}


\end{document}

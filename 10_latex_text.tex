% come fare un testo con latex;
% le funzioni vengono richiamate utilizzando il \nomefunzione e gli argomenti verranno passati con {};
% i commenti vengono eseguiti attraverso il simbolo percentuale;
% \documentclass{article} è la prima funzione utilizzata; successivamente c'è la funzione \usepackage{}, una funzione per aggiungere pacchetti aggiuntivi in latex;
% \begin{} è una funzione generica che inizia un processo in base all'argomento passato tra le {}
% \end{} termina il processo iniziato con la funzione \begin.
% \maketitle{} crea il titolo del file; 
% \section{} crea sezioni diverse del documento assegnandogli automaticamente un numero progressivo; scrivendo qualsiasi cosa dopo la funzione section (come un testo). 
% Se si manda a capo nel codice, compilandolo si avrà un'indentazione con il testo successivo alla riga vuota. Per togliere l'indentazione bisogna utilizzare la funzione \noindent

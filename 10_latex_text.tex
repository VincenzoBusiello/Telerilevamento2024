% come fare un testo con latex;
% le funzioni vengono richiamate utilizzando il \nomefunzione e gli argomenti verranno passati con {};
% i commenti vengono eseguiti attraverso il simbolo percentuale;
% \documentclass{article} è la prima funzione utilizzata; successivamente c'è la funzione \usepackage{}, una funzione per aggiungere pacchetti aggiuntivi in latex;
% \begin{} è una funzione generica che inizia un processo in base all'argomento passato tra le {}
% \end{} termina il processo iniziato con la funzione \begin.
% \maketitle{} crea il titolo del file; 
% \section{} crea sezioni diverse del documento assegnandogli automaticamente un numero progressivo; scrivendo qualsiasi cosa dopo la funzione section (come un testo). 
% Se si manda a capo nel codice, compilandolo si avrà un'indentazione con il testo successivo alla riga vuota. Per togliere l'indentazione bisogna utilizzare la funzione \noindent
\documentclass[12pt]{article}
\usepackage{graphicx} % Required for inserting images
\usepackage{hyperref}

\title{My first LaTeX doc}
\author{Vincenzo Busiello}
%\date{ }

\begin{document}

\maketitle
\tableofcontents

\section{Introduction} 
\label{sec:intro} % nel caso in cui si sbagli il nome della sezione compariranno dei ?? nella porzione in cui richiamo la \ref{}
E ho tutto dentro e poi mi accorgo che non ho parole
Non c'è poesia solo malinconia e malumore
E resto qui tra le mie mani e il resto del dolore
E resto qui tra le mie mani e una contraddizione
Me ne andrò controvento
Sarò via in un momento
Me ne andrò controvento
Sarò via in un momento

Ora vorrei spiazzare il tempo senza far rumore
Tutto và via mi lascia qui a guardarmi frantumare
Non c'è poesia nè voglia di aver timore
La cera brucia adesso non mi servono parole
E ho tutto dentro e poi mi accorgo che non ho parole
Non c'è poesia solo malinconia e malumore
E resto qui tra le mie mani e il resto del dolore
E resto qui tra le mie mani e una contraddizione
Me ne andrò controvento
Sarò via in un momento
Me ne andrò controvento
Sarò via in un momento

Ora vorrei spiazzare il tempo senza far rumore
Tutto và via mi lascia qui a guardarmi frantumare
Non c'è poesia nè voglia di aver timore
La cera brucia adesso non mi servono parole
E ho tutto dentro e poi mi accorgo che non ho parole
Non c'è poesia solo malinconia e malumore
E resto qui tra le mie mani e il resto del dolore
E resto qui tra le mie mani e una contraddizione
Me ne andrò controvento
Sarò via in un momento
Me ne andrò controvento
Sarò via in un momento
Ora vorrei spiazzare il tempo senza far rumore
Tutto và via mi lascia qui a guardarmi frantumare
Non c'è poesia nè voglia di aver timore
La cera brucia adesso non mi servono parole

\section{Methods}
\subsection{Study Area}
\subsection{Algorithms}
The equation used was Equation\ref{eq:sum}: 
\begin{equation}
    T = \sum p_i
    \label{eq:sum}
\end{equation}

In this thesis we made use of Equation \ref{eq:newton}:

\begin{equation}
    F=\sqrt[n]{G \frac{m_1 \times m_2}{d^2}}
    \label{eq:newton}
\end{equation}

\section{Results}


\section{Discussion}

Our results are in line with previous paper, introduce in section \ref{sec:intro}.

\end{document}

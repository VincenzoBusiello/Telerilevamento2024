\documentclass[12pt]{article}


\usepackage{graphicx} % Il pacchetto permette di caricare delle immagini
\usepackage{hyperref} % Permette di inserire collegamenti all'interno del testo
\usepackage{natbib} % permette di gestire la bibliografia 
\usepackage{lineno} % numera le righe del testo
\linenumbers % numera le righe del testo


\title{My first LaTeX doc}
\author{Vincenzo Busiello}
%\date{} % permette di cambiare l'impostazione della data a seconda degli argomenti che vengono passati. Se non è presente, pesca direttamente la data dal sistema impostandola come mm/gg/aa. 

\begin{document}

\maketitle
\tableofcontents

\section{Introduction} 
\label{sec:intro} % nel caso in cui si sbagli il nome della sezione compariranno dei ?? nella porzione in cui richiamo la \ref{}. Ciò permette di poter cercare eventuali errori all'interno del testo in modo più semplice. 
%\textbf{E ho tutto dentro e poi mi accorgo che non ho parole} % \textb mette in grassetto la parte di testo (che ho preso a caso da internet) passata come argomento
%\textit{Non c'è poesia solo malinconia e malumore} % \textit formatta il testo passato come argomento in corsivo

In this thesis we dealt with Dominance loss and tenure maintenance in Kalahari meerkats, one of the main themes in ethology \cite{Duncan2023}\\ 
%\smallskip % serve per inserire uno piccolo spazio tra un paragrafo e il successivo
We were very interested in developmental influences \cite{LaLoggia2024}
\bigskip{} % inserisce uno spazio più grande rispetto a \smallskip
The Tree-OH are the best band ever. 
\footnote{Source: Io e Carlo, ma anche Alfio <3} % inserisce la nota a pié di pagina


\section{Methods}
\subsection{Study Area}
\subsection{Algorithms}
The equation used was Equation\ref{eq:sum}: 
\begin{equation} 
    T = \sum p_i % \sum inserisce il simbolo della sommatoria
    \label{eq:sum}
\end{equation}

In this thesis we made use of Equation \ref{eq:newton}:

\begin{equation}
    F=\sqrt[n]{G \frac{m_1 \times m_2}{d^2}} % \sqrt = radice quadrata. tra le [] si può inserire l'esponente della radice. l'underscore serve per inserire il pedice. \times inserisce l'operatore moltiplicazione. 
    \label{eq:newton}
\end{equation}

\section{Results}
In this thesis the algorithm led to the results shown in figure \ref{fig:pulcinella}

\section{Discussion}
In this theses we demostrated that:
\begin{itemize}
    \item Dominance loss and tenure maintanance in Kalahari meerkats
    \item Developmental influences on Social Competence and Neuroplasticity
\end{itemize}

\begin{enumerate}
    \item Dominance loss and tenure maintanance in Kalahari meerkats
    \item Developmental influences on Social Competence and Neuroplasticity
\end{enumerate}
    
Our results are in line with previous paper, introduce in section \ref{sec:intro}.

\section{Data Availability}
All data  used in this thesis are available at \url{www.nasa.com}
\begin{thebibliography}{999}
    
    \bibitem[Duncan et al., 2023]{Duncan2023} 
    Duncan, C., Thorley, J., Manser, M. B., \& Clutton-Brock, T. (2023). Dominance loss and tenure maintenance in Kalahari meerkats. Behavioral Ecology, 34(6), 979-991.
    
    \bibitem[La Loggia, 2024]{LaLoggia2024}
    La Loggia, O. (2024). Developmental Influences on Social Competence and Neuroplasticity: The Impact of Early Social Complexity in a Cooperatively Breeding Fish (Doctoral dissertation, Institute of Ecology).

\end{thebibliography}

\newpage
\begin{figure}
    
    \centering % serve a centrare la figura in mezzo alla pagina
    \includegraphics[width=0.5\textwidth]{Pulcinella.jpg} % con il parametro \textwidth si adatta l'immagine alla grandezza del testo; moltiplicandola per 0.5 si ottiene l'immagine pari alla metà. 
    \caption{Enter Caption}
    \label{fig:pulcinella}
    
\end{figure}

\newpage
\hline
\bigskip
\textbf{Box 1 - blabla}
\bigskip
\hline
\bigskip
\begin{itemize} % serve a fare un elenco puntato
    \item The Tree-OH are the best band ever.
    \item The Tree-OH are the best band ever.
    \item The Tree-OH are the best band ever.
\end{itemize}
\bigskip
\hline
\end{document}
